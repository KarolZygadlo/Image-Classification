\documentclass[12pt,a4paper,twoside,openright]{report}
\usepackage{setspace} % importowanie pakietu do ustawiania interlinii
\onehalfspacing % ustawienie interlinii 1.5
\usepackage{tgtermes}
\usepackage[T1]{fontenc}
\usepackage{polski}
\usepackage[utf8]{inputenc}
\input glyphtounicode
\pdfgentounicode=1
\usepackage{amssymb}
\usepackage{amsmath}
\usepackage{float}
\usepackage{pgfplots}
\usepackage{titlesec}
\usepackage{pgf-pie}
\usepackage{floatrow}
\usepackage{sectsty}
\usepackage{placeins}
\usepackage{pdfpages}
\usepackage{csquotes}
\usepackage{graphicx}
\usepackage{caption}
\let\lll\undefined
\usepackage{indentfirst}
\usepackage{float}
\usepackage{tabularx}
\usepackage{hyperref}
\graphicspath{ {./img/} }
\usepackage[left=2.5cm,right=2.5cm,top=2.5cm,bottom=2.5cm]{geometry}
\usepackage{caption}
\captionsetup{font=small}

\newcommand{\mycaption}[2]{%
  \caption[#1]{#1\\ \footnotesize{#2}}%
}
\usepackage{listings,lipsum}
\lstnewenvironment{myverbatim}[1][]{%
  \lstset{
    basicstyle=\ttfamily,
    frame=tb,
    #1
  }%
}{}
\usepackage{tocloft}
\usepackage{etoolbox}
\usepackage{datetime}
\usepackage{dirtree}
\renewcommand\DTstyle{\rmfamily}
\usepackage{listings}
\usepackage{titlesec}
\usepackage[defernumbers=true,
        locallabelwidth,
        backend=biber]{biblatex}

\pretocmd{\chapter}{\addtocontents{toc}{\protect\addvspace{-5pt}}}{}{}

\newcommand*{\typeprefix}{%
  \ifentrytype{article}{%%
  }{%
    \ifentrytype{book}{%%
    }{%
      www. % "other"
    }%
  }%
}

\DeclareFieldFormat{labelnumber}{\typeprefix #1}
\renewcommand*{\finentrypunct}{}

\lstset{
	basicstyle=\footnotesize\ttfamily,
    keepspaces=true,
    showstringspaces=false,
    commentstyle=\ttfamily,
    frame=single,
    aboveskip=18pt,
    belowskip=18pt,
    breaklines=true,
}
\raggedbottom
\usepackage[justification=centering]{caption}
\makeatletter
% \def\@makechapterhead#1{%
% \pagebreak
%   \vspace*{120\p@}% <----------------- Space from top of page to Chapter #
%   {\parindent \z@ \raggedright
%     \ifnum \c@secnumdepth >\m@ne
%         \huge\bfseries \@chapapp\space \thechapter. \Huge \bfseries #1\par\nobreak% <-- Chapter #
%         \par\nobreak
%         \vskip 24\p@% <-------------- Space between Chapter # and title
%   }}

\titlespacing{\section}{0pt}{12pt}{6pt}
\titlespacing{\subsection}{0pt}{6pt}{6pt}
\titlespacing{\footnotesize}{0pt}{10pt}
% Ustawienie domyślnej wielkości czcionki dla sekcji na 13pt
\makeatletter
\renewcommand{\section}{\@startsection{section}{1}{0mm}{\baselineskip}{0.5\baselineskip}{\normalfont\fontsize{13}{15}\bfseries}}
\makeatother
% Ustawienie domyślnej wielkości czcionki dla tabel na 13pt
\AtBeginEnvironment{tabular}{\fontsize{10}{10}\selectfont}
\preto\tabular{\centering}
 % Formatowanie chaptera
\makeatletter
\def\@makechapterhead#1{%
  \vspace*{12\p@}% 
  {\parindent \z@ \raggedright \normalfont
    \ifnum \c@secnumdepth >\m@ne
        \fontsize{14pt}{18pt}\bfseries \@chapapp\space \thechapter.\space #1\par\nobreak
        \vskip 20\p@
    \fi
    \interlinepenalty\@M
    \vskip 6\p@ 
  }}
\makeatother


\author{Kamil Piech}
\linespread{1.5}
\let\cleardoublepage=\clearpage

\renewcommand\cftchapfont{\fontsize{12pt}{18pt}\bfseries}
\renewcommand\cftsecfont{\fontsize{12pt}{18pt}}
\renewcommand\cftchappagefont{\fontsize{12pt}{18pt}\bfseries}
\renewcommand\cftsecpagefont{\fontsize{12pt}{18pt}}
\renewcommand{\cfttoctitlefont}{\vspace{-1.5cm}\fontsize{14pt}{}\bfseries}

\begin{document}
\captionsetup{font=footnotesize}
    \begin{titlepage}
    \includepdf{TitlePage.pdf}
    \end{titlepage}
    \tableofcontents
\begingroup
\setlength{\parindent}{0.7cm}

    % \let\clearpage\relax
    \chapter{Wstęp}
\section{Cel i motywacja pracy}

Celem niniejszej pracy magisterskiej jest opracowanie i zastosowanie metod uczenia głębokiego w celu rozpoznawania wad produkcyjnych na podstawie analizy zdjęć. Ze względu na coraz większe zapotrzebowanie na szybkie i efektywne rozwiązania w zakresie kontroli jakości, zastosowanie technik sztucznej inteligencji staje się niezbędne dla przemysłu. W szczególności, wykorzystanie uczenia głębokiego, jako jednego z najbardziej zaawansowanych podejść w dziedzinie sztucznej inteligencji, pozwala na znaczące zwiększenie skuteczności wykrywania wad produkcyjnych.

Motywacją do podjęcia tego tematu była chęć eksploracji i zrozumienia nowoczesnych technologii związanych z uczeniem maszynowym oraz ich praktyczne zastosowanie w kontekście przemysłowym. Wadliwe komponenty mogą prowadzić do znacznych strat finansowych oraz wpływać negatywnie na reputację przedsiębiorstwa. Automatyczne wykrywanie wad na wczesnym etapie procesu produkcyjnego może przyczynić się do zwiększenia efektywności, zmniejszenia kosztów oraz ograniczenia ilości produktów o niskiej jakości trafiających do odbiorców.

W ramach pracy magisterskiej opracowany zostanie system, który będzie w stanie analizować zdjęcia przedmiotów i automatycznie klasyfikować je jako wadliwe lub prawidłowe. System ten będzie oparty na technikach uczenia głębokiego, takich jak konwolucyjne sieci neuronowe, które są obecnie szeroko stosowane w różnych dziedzinach analizy obrazów. Praca będzie obejmować zarówno teorię, jak i praktyczne aspekty projektowania, implementacji oraz oceny tego rodzaju systemów.

W celu zilustrowania koncepcji i metod przedstawionych w pracy, zostanie przeprowadzone eksperymentalne zastosowanie opracowanego systemu do wybranego zbioru zdjęć reprezentujących obiekty z wadami produkcyjnymi oraz prawidłowymi komponentami. Wyniki tego eksperymentu posłużą jako dowód na skuteczność zastosowanego podejścia oraz jako punkt wyjścia do dalszej dyskusji na temat potencjalnych ulepszeń i przyszłych kierunków rozwoju.

\section{Zawartość pracy}

W niniejszej pracy dyplomowej skupiamy się na zagadnieniu wykrywania wad produkcyjnych z wykorzystaniem uczenia głębokiego, w szczególności sieci konwolucyjnych. Zakres pracy obejmuje następujące aspekty:

\begin{itemize}
\item Zaprezentowanie wybranych podstaw teoretycznych związanych z uczeniem głębokim, sieciami konwolucyjnymi oraz ich zastosowaniem w przemyśle;
\item Analiza istniejących rozwiązań i technologii stosowanych w kontroli jakości, w tym systemów wizyjnych, Internetu Rzeczy (IoT) oraz analizy danych;
\item Opracowanie projektu systemu, który spełnia określone wymagania funkcjonalne i niefunkcjonalne, oraz omówienie kroków projektowania takich jak wybór sieci konwolucyjnej, wczytywanie danych czy wstępne przetwarzanie danych;
\item Implementacja algorytmu uczenia głębokiego na podstawie opracowanego projektu, przedstawienie budowy modelu sieci neuronowej oraz etapów trenowania i testowania modelu;
\item Ewaluacja i analiza wyników uzyskanych podczas testów skuteczności i wydajności opracowanego systemu;
\item Prezentacja działania programu na przykładach zastosowań;
\item Podsumowanie wyników pracy, wskazanie zrealizowanych celów oraz przedstawienie perspektyw na dalsze badania i rozwój systemu.
\end{itemize}

W ramach pracy został opracowany kod źródłowy, który prezentuje implementację modelu sieci neuronowej oraz wszystkie etapy przetwarzania danych, trenowania, testowania i ewaluacji modelu. Kod źródłowy został napisany w wybranym języku programowania i korzysta z odpowiednich bibliotek oraz narzędzi dedykowanych dla uczenia głębokiego i sieci konwolucyjnych.
    \chapter{Wybrane podstawy teoretyczne pracy}

\section{Uczenie głębokie}
Uczenie głębokie to gałąź uczenia maszynowego, która skupia się na zastosowaniu sztucznych sieci neuronowych z wieloma warstwami ukrytymi. W przeciwieństwie do tradycyjnych metod uczenia maszynowego, takich jak regresja liniowa czy drzewa decyzyjne, uczenie głębokie automatycznie odkrywa reprezentacje danych na różnych poziomach abstrakcji. Jest to kluczowe dla rozwiązywania złożonych problemów, takich jak rozpoznawanie obrazów, przetwarzanie języka naturalnego, rozpoznawanie mowy czy analiza danych biologicznych. 
Uczenie głębokie to specjalna odmiana uczenia maszynowego, w ramach której sztuczne sieci neuronowe, czyli algorytmy stworzone tak, aby naśladować sposób działania ludzkiego mózgu, uczą się na podstawie obszernych zbiorów danych. Uczenie głębokie opiera się na wielowarstwowych sieciach neuronowych, które są algorytmami wzorowanymi na sposób funkcjonowania ludzkich mózgów. Szkolenie z dużą ilością danych kształtuje neurony w sieci neuronowej, co prowadzi do stworzenia modelu uczenia głębokiego zdolnego do przetwarzania nowych danych po przeszkoleniu. Modele uczenia głębokiego korzystają z informacji z różnych źródeł danych i analizują je w czasie rzeczywistym, bez potrzeby ingerencji człowieka. W procesie uczenia głębokiego, procesory graficzne (GPU) są optymalizowane do pracy z modelami szkoleniowymi, ponieważ mogą równocześnie wykonywać wiele obliczeń.
Uczenie głębokie jest siłą napędową wielu technologii sztucznej inteligencji (AI), które mogą zautomatyzować i usprawnić zadania analityczne. Większość osób codziennie styka się z uczeniem głębokim, przeglądając internet czy korzystając z telefonów komórkowych. Wśród licznych zastosowań, uczenie głębokie jest wykorzystywane do generowania napisów dla filmów na YouTube, rozpoznawania mowy w telefonach i inteligentnych głośnikach, identyfikowania twarzy na zdjęciach czy tworzenia autonomicznych pojazdów. W miarę jak analitycy i badacze danych podejmują coraz bardziej zaawansowane projekty związane z uczeniem głębokim, wykorzystując architektury głębokich sieci neuronowych, ten rodzaj sztucznej inteligencji stanie się jeszcze bardziej powszechny w naszym codziennym życiu.


\subsection{Historia uczenia głębokiego}
Historia uczenia głębokiego sięga lat 40. XX wieku, kiedy to badacze zaczęli eksplorować koncepcje sztucznych neuronów, które próbowały naśladować biologiczne procesy zachodzące w ludzkim mózgu. W 1986 roku Rumelhart, Hinton i Williams wprowadzili algorytm wstecznej propagacji błędów (backpropagation), który umożliwił efektywne uczenie się sieci neuronowych.
\paragraph{}
W 2006 roku Hinton i Salakhutdinov opublikowali pracę, która przyczyniła się do powstania współczesnego uczenia głębokiego. Przedstawili w niej skonstruowane przez siebie głębokie sieci wstępnie uczące się (deep belief networks) i pokazali, że sieci te potrafią uczyć się reprezentacji danych na wielu poziomach abstrakcji. Od tego czasu uczenie głębokie zyskało na popularności, a rozwój technologii przyczynił się do udoskonalenia algorytmów oraz wykorzystania uczenia głębokiego w praktyce.

\subsection{Zastosowania uczenia głębokiego}
Uczenie głębokie znalazło szerokie zastosowanie w różnych dziedzinach nauki i przemysłu. Przykłady obejmują:
\begin{itemize}
\item Rozpoznawanie obrazów: wykorzystanie sieci konwolucyjnych (CNN) do klasyfikacji obrazów, detekcji obiektów czy segmentacji obrazów.
\item Przetwarzanie języka naturalnego: stosowanie rekurencyjnych sieci neuronowych (RNN) i mechanizmów uwagi (attention) do tłumaczenia maszynowego, generowania tekstu czy analizy uczuć.
\item Rozpoznawanie mowy: zastosowanie sieci neuronowych do rozpoznawania mowy i konwersji mowy na tekst.
\item Wzmocnienie ucznia: wykorzystanie uczenia głębokiego w połączeniu z algorytmami uczenia przez wzmacnianie do sterowania robotami, pojazdami autonomicznymi czy strategiami giełdowymi.
\end{itemize}

\subsection{Różnice między uczeniem głębokim a innymi technikami uczenia maszynowego}
Uczenie głębokie różni się od innych technik uczenia maszynowego w kilku kluczowych aspektach:

\begin{itemize}
\item \textbf{Reprezentacja danych:} Uczenie głębokie automatycznie uczy się reprezentacji danych na różnych poziomach abstrakcji, podczas gdy w tradycyjnych metodach uczenia maszynowego, takich jak SVM czy drzewa decyzyjne, inżynierowie muszą ręcznie projektować cechy (feature engineering), które mają być użyte do uczenia modelu.
\item \textbf{Architektura sieci:} Sieci neuronowe stosowane w uczeniu głębokim mają wiele warstw ukrytych, co pozwala na uczenie się bardziej złożonych funkcji. W przeciwnym razie, metody uczenia maszynowego często korzystają z modeli o mniejszej złożoności, takich jak regresja liniowa czy drzewa decyzyjne.
\item \textbf{Skalowalność:} Dzięki efektywnym algorytmom optymalizacji oraz rosnącej mocy obliczeniowej, uczenie głębokie może być stosowane do przetwarzania ogromnych zbiorów danych. W porównaniu, inne techniki uczenia maszynowego mają trudności z działaniem na dużą skalę, szczególnie gdy wymagane jest ekstrahowanie cech z dużych zbiorów danych.
\item \textbf{Transfer wiedzy:} W uczeniu głębokim istnieje możliwość wykorzystania wiedzy uzyskanej z jednego zadania do innych zadań, co jest nazywane uczeniem transferowym (transfer learning). W tradycyjnym uczeniu maszynowym transfer wiedzy jest trudniejszy do osiągnięcia.
\end{itemize}

\section{Sieci konwolucyjne}
Sieci konwolucyjne (CNN) to specjalny rodzaj sieci neuronowych, w których zastosowano warstwy konwolucyjne. Warstwy te analizują obrazy za pomocą filtrów, które są przesuwane po danych wejściowych, generując mapy cech. Filtry te uczą się wykrywać lokalne wzorce, takie jak krawędzie, tekstury czy kształty. Warstwy pooling (agregujące) są stosowane w celu zmniejszenia wymiarowości map cech, co prowadzi do redukcji ilości parametrów w sieci. Sieci konwolucyjne mogą być również wykorzystywane w połączeniu z innymi warstwami, takimi jak warstwy gęsto połączone (fully connected) czy rekurencyjne (RNN), w zależności od problemu, który mają rozwiązać. Sieci neuronowe konwolucyjne funkcjonują przez gromadzenie i przetwarzanie dużych zbiorów danych w postaci siatek, a następnie ekstrakcję kluczowych cech szczegółowych w celu klasyfikacji i rozpoznawania. CNN zazwyczaj składają się z trzech głównych typów warstw: konwolucyjnej, buforowej (pooling) i w pełni połączonej. Każda z tych warstw pełni inną funkcję, wykonuje określone zadanie na zebranych danych oraz uczy się coraz bardziej złożonych aspektów.

\subsection{Jak opracowywane są sieci CNN?}
Sieci neuronowe konwolucyjne (CNN) odgrywają kluczową rolę w uczeniu głębokim i umożliwiają szerokie zastosowania w różnorodnych sektorach na całym świecie. Aby w pełni zrozumieć ich znaczenie, ważne jest, aby zrozumieć sposób ich tworzenia. Tworzenie CNN to czasochłonny i skomplikowany proces, który obejmuje trzy etapy: uczenie, optymalizację oraz wnioskowanie. Intel współpracuje bezpośrednio z programistami i analitykami danych w celu opracowania innowacyjnych metod poprawy i przyspieszenia tego procesu, co pozwala na szybsze i łatwiejsze wdrażanie nowych rozwiązań.
\subsubsection{Szkolenia}
Szkolenie (trening) w konwolucyjnych sieciach neuronowych (CNN) to proces, w którym sieć uczy się rozpoznawać wzorce i cechy na podstawie danych treningowych. Celem tego procesu jest optymalizacja wag sieci neuronowej, tak aby była w stanie dokonywać poprawnych predykcji na nowych, wcześniej niewidzianych danych. Proces szkolenia CNN składa się z kilku kroków:
\begin{itemize}
    \item \textbf{Inicjalizacja wag:} Na początek, wagi sieci neuronowej są inicjalizowane losowo lub za pomocą specjalnych technik inicjalizacji.
    \item \textbf{Przekazanie danych do sieci (propagacja wprzód):} Dane treningowe, takie jak obrazy, są przekazywane przez sieć neuronową. W trakcie tego procesu dane przechodzą przez różne warstwy sieci, takie jak warstwy konwolucyjne, warstwy buforowe (pooling) oraz warstwy w pełni połączone. W każdej warstwie sieć ekstrahuje cechy z danych wejściowych, które są przekazywane do kolejnych warstw.
    \item \textbf{Obliczenie funkcji straty:} Na końcu sieci neuronowej obliczana jest wartość funkcji straty, która mierzy różnicę między predykcjami sieci a rzeczywistymi etykietami danych treningowych. Funkcja straty jest kluczowym elementem procesu uczenia, ponieważ określa, jak dobrze sieć radzi sobie z zadaniem.
    \item \textbf{Propagacja wsteczna (Backpropagation): }Po obliczeniu funkcji straty, gradienty tej funkcji są obliczane wstecznie dla każdej warstwy sieci. Gradienty te informują o kierunku, w którym wagi sieci powinny zostać zaktualizowane, aby zmniejszyć wartość funkcji straty.
    \item \textbf{Aktualizacja wag:} Wagi sieci neuronowej są aktualizowane za pomocą algorytmu optymalizacji, takiego jak stochastyczny spadek gradientu (SGD), Adam czy RMSprop. Algorytm optymalizacji stosuje obliczone gradienty do wag sieci, modyfikując je w taki sposób, aby funkcja straty była mniejsza.
    \item \textbf{Iteracje i epoki: }Powyższe kroki są powtarzane wielokrotnie dla całego zbioru danych treningowych. Przejście przez cały zbiór danych to jedna epoka. Proces szkolenia zwykle obejmuje wiele epok, aż sieć osiągnie zadowalający poziom dokładności.
\end{itemize}
Po zakończeniu procesu szkolenia, CNN jest gotowa do przeprowadzania wnioskowania na nowych danych. Wagi sieci zostały zoptymalizowane, dzięki czemu potrafi ona dokonywać poprawnych predykcji na podstawie cech wykrytych w danych wejściowych.
\subsubsection{Optymalizacja}
Optymalizacja w konwolucyjnych sieciach neuronowych (CNN) polega na dostosowywaniu parametrów sieci w celu zminimalizowania funkcji kosztu i poprawy wydajności sieci. Proces optymalizacji prowadzi do lepszego uczenia się przez sieć istotnych cech danych wejściowych, co przekłada się na lepsze wyniki predykcji.
Optymalizacja w sieciach CNN obejmuje kilka aspektów:
\begin{itemize}
    \item \textbf{Aktualizacja wag:} Optymalizacja polega na dostosowywaniu wag sieci neuronowej w odpowiedzi na błędy popełniane podczas procesu uczenia. Metody aktualizacji wag, takie jak stochastyczny gradient prosty (SGD) czy adaptacyjne metody optymalizacji (np. Adam, RMSprop), wykorzystują informacje o gradientach funkcji kosztu do aktualizacji wag.
    \item \textbf{Dobór hiperparametrów:} Optymalizacja może obejmować dobór odpowiednich hiperparametrów, takich jak liczba warstw, liczba neuronów w każdej warstwie, współczynnik uczenia się, wielkość batcha czy funkcje aktywacji. Hiperparametry można dobrać za pomocą technik takich jak przeszukiwanie siatki, przeszukiwanie losowe czy optymalizacja bayesowska.
    \item \textbf{Regularyzacja:} Wprowadzenie technik regularyzacji, takich jak L1, L2 czy dropout, może pomóc w zapobieganiu nadmiernemu dopasowaniu sieci, co pozytywnie wpływa na jej zdolność generalizacji.
    \item \textbf{Architektura sieci:} Optymalizacja obejmuje również eksperymentowanie z różnymi architekturami sieci, aby znaleźć taką, która najlepiej radzi sobie z konkretnym zadaniem. Warto zbadać istniejące architektury, które odniosły sukces w podobnych problemach, takie jak VGG, ResNet czy Inception.
    \item \textbf{Techniki augmentacji danych:} Optymalizacja może również obejmować zastosowanie technik augmentacji danych, które polegają na wprowadzeniu różnych transformacji na danych wejściowych, takich jak obracanie, przeskalowanie, odbicie lustrzane czy zmiana oświetlenia. Augmentacja danych pomaga sieci lepiej generalizować na nowych danych.
\end{itemize}
\subsubsection{Wnioskowanie}
Wnioskowanie (ang. inference) w konwolucyjnych sieciach neuronowych (CNN) odnosi się do procesu, w którym wcześniej wytrenowana sieć jest stosowana do analizy nowych danych wejściowych i generowania predykcji. W przypadku CNN, wnioskowanie często dotyczy zastosowań takich jak klasyfikacja obrazów, detekcja obiektów czy segmentacja semantyczna.
W procesie wnioskowania, dane wejściowe, takie jak obraz, są przekazywane przez sieć CNN, która przeprowadza szereg operacji w różnych warstwach. Warstwy te obejmują warstwy konwolucyjne, warstwy buforowe (pooling) oraz warstwy w pełni połączone. Każda warstwa analizuje dane wejściowe i ekstrahuje istotne cechy, które są przekazywane do kolejnych warstw sieci.
W trakcie wnioskowania sieć neuronowa korzysta z wcześniej nauczonej hierarchii cech oraz wag, które zostały optymalizowane podczas procesu treningu. Dzięki tym wagom sieć jest w stanie rozpoznać i klasyfikować nowe dane wejściowe.
Na końcu sieci znajduje się warstwa wyjściowa, która generuje wynik końcowy, czyli predykcję. W przypadku klasyfikacji obrazów może to być prawdopodobieństwo przynależności obrazu do danej klasy. W detekcji obiektów sieć może generować współrzędne prostokątów otaczających obiekty oraz ich kategorie.
Wnioskowanie jest zwykle znacznie szybsze niż proces uczenia, ponieważ nie wymaga obliczania gradientów ani aktualizacji wag. Jako że wagi są już wcześniej nauczone, sieć koncentruje się na przeprowadzeniu operacji na nowych danych wejściowych, aby wygenerować predykcję. W przypadku aplikacji w czasie rzeczywistym, takich jak rozpoznawanie mowy czy analiza wideo, szybkość wnioskowania jest kluczowym czynnikiem wpływającym na użyteczność i skuteczność sieci neuronowej.
\section{Wady produkcyjne i ich rodzaje}
Wady produkcyjne to niezgodności w produkcji, które powodują, że produkt nie spełnia swoich wymagań jakościowych. Mogą być spowodowane przez wiele czynników, takich jak nieprawidłowości w procesie produkcyjnym, zużycie maszyn czy błędy ludzkie. Wady produkcyjne można podzielić na różne rodzaje, w zależności od charakterystyki produktu i procesu produkcyjnego. Niektóre z najbardziej powszechnych rodzajów wad produkcyjnych obejmują:

\begin{itemize}
\item \textbf{Wady powierzchniowe:} Pęknięcia, zadrapania, zmarszczki, przebarwienia czy zanieczyszczenia na powierzchni produktu.
\item \textbf{Wady geometryczne:} Niewłaściwe wymiary, kształty czy położenie elementów w stosunku do siebie nawzajem.
\item \textbf{Wady materiałowe:} Wady w strukturze materiału, takie jak pęcherze powietrza, porowatość czy zanieczyszczenia.
\item \textbf{Wady funkcjonalne:} Problemy z działaniem produktu wynikające z nieprawidłowego montażu, błędów w projektowaniu czy wadliwych komponentów.
\end{itemize}

\section{Istniejące rozwiązania}
W przemyśle istnieje wiele rozwiązań do wykrywania wad produkcyjnych. Tradycyjne metody obejmują wizualne sprawdzanie produktów przez operatorów, które może być czasochłonne i podatne na błędy. Inne metody obejmują stosowanie systemów wizyjnych z kamerami i algorytmami przetwarzania obrazów, które analizują produkty pod kątem wad. Te metody jednak często wymagają ręcznego projektowania cech oraz są trudne do skalowania.

Uczenie głębokie, a w szczególności sieci konwolucyjne, oferuje nowe możliwości w zakresie wykrywania wad produkcyjnych. Dzięki automatycznemu uczeniu się reprezentacji danych, sieci konwolucyjne potrafią wykrywać wady na obrazach z wysoką precyzją i są łatwe do skalowania. Ponadto, uczenie transferowe pozwala na zastosowanie wiedzy uzyskanej z jednego zadania do innego, co ułatwia adaptację modeli do różnych rodzajów wad i procesów produkcyjnych.

\section{Wybrane języki programowania}
\subsubsection{Python}
Python to popularny język programowania o otwartym kodzie źródłowym, który cechuje się prostotą, czytelnością i wszechstronnością. Jest szeroko stosowany w uczeniu maszynowym i uczeniu głębokim dzięki bogatemu ekosystemowi bibliotek, takich jak TensorFlow, PyTorch czy scikit-learn. Python oferuje wiele narzędzi do analizy danych, wizualizacji, pracy z obrazami i dźwię

    \chapter{Wymagania funkcjonalne oraz niefunkcjonalne}

\section{Wymagania funkcjonalne}
Wymagania funkcjonalne systemu obejmują następujące elementy:

\begin{enumerate}
\item \textbf{Wczytywanie i przetwarzanie danych wejściowych (obrazów):} System powinien być w stanie wczytać dane wejściowe w postaci obrazów oraz przetworzyć je w celu przygotowania do analizy przez model.

\item \textbf{Trenowanie modelu na zbiorze treningowym:} System powinien być wyposażony w model oparty na uczeniu głębokim, który jest trenowany na zbiorze treningowym, zawierającym obrazy uszkodzonych i prawidłowych elementów.

\item \textbf{Walidacja modelu na zbiorze walidacyjnym:} System powinien wykorzystywać zbiór walidacyjny, aby sprawdzić jakość modelu w trakcie procesu trenowania. Walidacja pozwala dostosować hiperparametry modelu, aby uniknąć nadmiernego dopasowania (overfitting).

\item \textbf{Testowanie modelu na zbiorze testowym:} Po zakończeniu trenowania, system powinien zostać przetestowany na zbiorze testowym, który zawiera obrazy nieznane dla modelu. Wyniki testów pozwolą ocenić ostateczną jakość modelu.

\item \textbf{Klasyfikacja obrazów na części uszkodzone i prawidłowe:} Głównym celem systemu jest klasyfikacja obrazów na części uszkodzone i prawidłowe, co pozwala zidentyfikować problemy z jakością w procesie produkcyjnym.

\end{enumerate}

\section{Wymagania niefunkcjonalne}
Wymagania niefunkcjonalne systemu odnoszą się do cech jakościowych, takich jak:

\begin{enumerate}

\item \textbf{Dokładność klasyfikacji:} System powinien osiągać wysoką dokładność klasyfikacji, aby skutecznie identyfikować uszkodzone i prawidłowe elementy.

\item \textbf{Czas uczenia modelu:} Czas trenowania modelu powinien być na tyle krótki, aby umożliwić szybkie dostosowanie modelu do nowych danych.

\item \textbf{Złożoność obliczeniowa modelu:} Model powinien być na tyle prosty, aby możliwe obliczenia nie obciążały nadmiernie zasobów sprzętowych, jednocześnie zachowując wysoką jakość klasyfikacji.

\item \textbf{Skalowalność systemu:} System powinien być skalowalny, co oznacza, że powinien być w stanie obsłużyć większe ilości danych oraz dostosować się do zmieniających się warunków (np. dodanie nowych klas obiektów do klasyfikacji).

\item \textbf{Współczynnik fałszywych pozytywów i fałszywych negatywów:} System powinien charakteryzować się niskim współczynnikiem fałszywych pozytywów (FP) i fałszywych negatywów (FN). Fałszywe pozytywy to przypadki, gdy system błędnie klasyfikuje uszkodzone elementy jako prawidłowe, natomiast fałszywe negatywy to błędna klasyfikacja prawidłowych elementów jako uszkodzone. Oba te rodzaje błędów mogą prowadzić do niekorzystnych skutków, takich jak przestój w produkcji, czy też przekroczenie progów jakościowych.

\end{enumerate}

\chapter{Projekt rozwiązania}

\section{Wybór sieci konwolucyjnej}
Sieci konwolucyjne są szczególnie odpowiednie dla problemów związanych z analizą obrazów, ponieważ potrafią automatycznie uczyć się cech na różnych poziomach abstrakcji. W przeciwnym razie, ręczne projektowanie cech obrazu, zwłaszcza w przypadku złożonych zadań klasyfikacji, może być żmudne i czasochłonne. CNN pozwala nam wykryć lokalne wzorce w obrazach, takie jak kształty i tekstury, które są istotne dla zadania klasyfikacji.

\section{Korzyści z zastosowania uczenia głębokiego}
Uczenie głębokie, jako poddziedzina uczenia maszynowego, oferuje wiele korzyści w kontekście klasyfikacji obrazów. Oto niektóre z nich:

\begin{itemize}
\item \textbf{Automatyczna ekstrakcja cech:} Uczenie głębokie pozwala na automatyczne wykrywanie istotnych cech w danych, eliminując potrzebę ręcznego projektowania cech. W przypadku klasyfikacji obrazów oznacza to, że sieci głębokie potrafią uczyć się hierarchicznych reprezentacji obrazów, co prowadzi do lepszej wydajności klasyfikacji.

\item \textbf{Generalizacja:} Uczenie głębokie ma zdolność do generalizacji na nowe, niewidziane wcześniej dane. Oznacza to, że model wytrenowany na odpowiednio dużym i różnorodnym zbiorze danych może skutecznie klasyfikować obrazy, które nie były częścią jego zbioru treningowego.

\item \textbf{Skalowalność:} Architektury uczenia głębokiego są elastyczne i łatwo skalowalne. Można je dostosować do różnych rozmiarów i rodzajów danych, co pozwala na efektywne rozwiązanie problemów o różnym stopniu złożoności.

\item \textbf{Wydajność:} Dzięki zastosowaniu akceleratorów sprzętowych, takich jak GPU, proces uczenia głębokich sieci neuronowych można znacznie przyspieszyć, co prowadzi do szybszego rozwoju i wdrożenia modeli.

\end{itemize}

\section{Ogólny zarys rozwiązania}
Zaprogramowane rozwiązanie będzie oparte na sieci konwolucyjnej (CNN), która będzie trenowana na danych obrazowych przedstawiających uszkodzone i prawidłowe elementy. Przygotowany model będzie następnie testowany na zbiorze testowym w celu oceny jego dokładności oraz zdolności generalizacji.

W kolejnych podrozdziałach przedstawione zostną szczegółowe etapy procesu projektowania rozwiązania, takie jak:

\begin{itemize}
\item Import bibliotek
\item Wczytywanie danych
\item Ładowanie danych
\item Wstępne przetwarzanie danych
\item Podział danych na zestawy treningowe, walidacyjne i testowe
\end{itemize}

Na podstawie uzyskanych wyników będzie można ocenić jakość opracowanego rozwiązania oraz jego potencjalne ograniczenia. 

\section{Import bibliotek}
W tym podrozdziale omówione zostaną wykorzystane biblioteki oraz ich zastosowanie w projekcie. W naszym rozwiązaniu korzystamy z następujących bibliotek:

\begin{itemize}
\item \textbf{TensorFlow:} Główna biblioteka do uczenia głębokiego, która umożliwia definiowanie, trenowanie i ewaluację modeli sieci neuronowych. W projekcie wykorzystany jest TensorFlow do implementacji sieci konwolucyjnej (CNN) oraz zarządzania procesem uczenia i walidacji modelu.

\item \textbf{OpenCV:} Popularna biblioteka do przetwarzania obrazów, która umożliwia wczytywanie, modyfikowanie i zapisywanie obrazów w różnych formatach. W projekcie wykorzystany jest OpenCV do wczytywania obrazów z dysku, ich przekształceń (np. skalowanie, obrót), a także sprawdzania poprawności danych obrazowych.

\item \textbf{Matplotlib:} Biblioteka do generowania wysokiej jakości wykresów 2D i 3D. W projekcie wykorzystany jest Matplotlib do wizualizacji danych, takich jak wykresy dokładności i straty w trakcie procesu uczenia oraz prezentacji wyników klasyfikacji obrazów.

\item \textbf{NumPy:} Biblioteka do obsługi macierzy wielowymiarowych, która oferuje wiele funkcji matematycznych o dużej wydajności. W projekcie wykorzystany jest NumPy do manipulacji danymi obrazowymi, przetwarzania macierzy oraz realizacji obliczeń matematycznych.

\end{itemize}

Do pracy z tymi bibliotekami należy je zaimportować, korzystając z poniższego kodu:

\begin{lstlisting}[language=Python]
import tensorflow as tf
import cv2
import matplotlib.pyplot as plt
import numpy as np
\end{lstlisting}

Dzięki temu mamy dostęp do wszystkich funkcji i klas oferowanych przez te biblioteki, co pozwala na efektywną realizację projektu.

Kolejnym krokiem jest konfiguracja pamięci GPU, aby uniknąć wykorzystania całej dostępnej pamięci przez TensorFlow. Dzięki temu inni użytkownicy również będą mogli korzystać z GPU.

\begin{lstlisting}[language=Python]
import tensorflow as tf
import os

gpus = tf.config.experimental.list_physical_devices('GPU')
for gpu in gpus:
tf.config.experimental.set_memory_growth(gpu, True)

tf.config.list_physical_devices('GPU')
\end{lstlisting}

\section{Wczytywanie danych}
W celu wczytania danych do modelu TensorFlow, należy zacząć od zorganizowania katalogów ze zdjęciami. Klasyfikacja obrazów odbywa się na podstawie nazw folderów, gdzie każdy folder odpowiada jednej klasie. W opisywany, przypadku, są dwa foldery: 'uszkodzony' i 'prawidłowy'.

Podczas wczytywania danych, należy skontrolować jakość zdjęć, sprawdzając, czy można je załadować do biblioteki OpenCV oraz czy ich rozszerzenie jest zgodne z oczekiwanymi (np. JPEG, JPG, BMP, PNG). W przypadku wykrycia uszkodzonych lub nieobsługiwanych obrazów, są one usuwane. Poniżej przedstawiono fragment kodu odpowiedzialny za kontrolę i usuwanie wadliwych zdjęć:

\begin{lstlisting}[language=Python]
for image_class in os.listdir(data_dir):
for image in os.listdir(os.path.join(data_dir, image_class)):
image_path = os.path.join(data_dir, image_class, image)
try:
img = cv2.imread(image_path)
tip = imghdr.what(image_path)
if tip not in image_exts:
print('Zdjecie posiada nieobslugiwane rozszerzenie {}'.format(image_path))
os.remove(image_path)
except Exception as e:
print('Wystapil problem ze zdjeciem {}'.format(image_path))
\end{lstlisting}

\section{Ładowanie danych}
W kolejnym kroku trzeba załadować dane za pomocą funkcji \texttt{image\_dataset\_from\_directory} z biblioteki TensorFlow. Funkcja ta automatycznie wczytuje zdjęcia z katalogów i przetwarza je do odpowiedniego formatu. Następnie konwertujemy dane na iterator, aby uzyskać dostęp do poszczególnych zdjęć i ich etykiet.

\begin{lstlisting}[language=Python]
import numpy as np
from matplotlib import pyplot as plt

data = tf.keras.utils.image_dataset_from_directory('data')
data_iterator = data.as_numpy_iterator()
batch = data_iterator.next()

fig, ax = plt.subplots(ncols=4, figsize=(20,20))
for idx, img in enumerate(batch[0][:4]):
ax[idx].imshow(img.astype(int))
ax[idx].title.set_text(batch[1][idx])
\end{lstlisting}

\section{Wstępne przetwarzanie danych}
Wstępne przetwarzanie danych polega na przygotowaniu danych wejściowych, tak aby model uczenia głębokiego mógł lepiej zrozumieć i wykorzystać informacje zawarte w tych danych. W przypadku tego projektu, wstępne przetwarzanie danych obejmuje przeskalowanie wartości pikseli do zakresu 0-1, co ułatwia uczenie się modelu.

Przeskalowanie wartości pikseli polega na podzieleniu wartości każdego piksela przez 255, co jest najwyższą wartością możliwą dla wartości piksela w formacie RGB. Wynik tego przekształcenia daje wartości zmiennoprzecinkowe w zakresie od 0 do 1, co jest preferowanym zakresem wartości dla wielu algorytmów uczenia głębokiego, w tym również dla sieci neuronowych.

W poniższym fragmencie kodu przeskalowanie wartości pikseli jest wykonywane na całym zbiorze danych za pomocą funkcji \texttt{map}. Funkcja ta przekształca dane przy użyciu podanej funkcji, w tym przypadku przeskalowując wartości pikseli dzieląc przez 255.

\begin{lstlisting}[language=Python]
data = data.map(lambda x, y: (x / 255, y))
data.as_numpy_iterator().next()
\end{lstlisting}

Przeskalowanie wartości pikseli ma kilka zalet. Po pierwsze, przekształcone wartości są mniejsze, co może przyspieszyć proces uczenia się modelu. Po drugie, przeskalowanie wartości pikseli może również przyczynić się do poprawy zdolności generalizacji modelu, ponieważ wartości w podobnym zakresie mogą być łatwiej porównywalne przez sieć neuronową.

\section{Podział danych na zestawy treningowe, walidacyjne i testowe}
Podział danych na zestawy treningowe, walidacyjne i testowe jest kluczowym elementem procesu uczenia się modelu głębokiego. Różne zestawy danych mają różne funkcje w procesie uczenia się:

\begin{itemize}
\item \textbf{Zestaw treningowy} służy do uczenia modelu. W trakcie uczenia model aktualizuje swoje wagi na podstawie błędów, które popełnia w prognozowaniu etykiet dla danych treningowych.
\item \textbf{Zestaw walidacyjny} służy do monitorowania postępów modelu podczas uczenia. Na podstawie wyników na danych walidacyjnych, możemy dostosować parametry modelu, takie jak liczba epok, funkcje kosztu czy hiperparametry optymalizatora. Zestaw walidacyjny pomaga również w wykrywaniu problemów, takich jak nadmierne dopasowanie (overfitting).
\item \textbf{Zestaw testowy} służy do oceny końcowej wydajności modelu po zakończeniu uczenia. Wyniki na danych testowych dają nam informacje na temat zdolności generalizacji modelu do danych, których nie widział wcześniej.
\end{itemize}

Aby podzielić dane na odpowiednie zestawy, najpierw należy określić ich rozmiary. W tym przypadku, 70\% danych zostanie przeznaczone na trening, 20\% na walidację i 10\% na testowanie. Następnie trzeba skorzystać z metod \texttt{take}, \texttt{skip} i \texttt{take} dostępnych dla obiektu dataset w TensorFlow, aby wydzielić odpowiednie części danych.

\begin{lstlisting}[language=Python]
train_size = int(len(data).7)
val_size = int(len(data).2) + 1
test_size = int(len(data)*.1) + 1

train = data.take(train_size)
val = data.skip(train_size).take(val_size)
test = data.skip(train_size + val_size).take(test_size)
\end{lstlisting}

Warto zauważyć, że podział danych może być również wykonywany za pomocą funkcji \texttt{train\_test\_split} z biblioteki scikit-learn, jednak w tym przypadku korzystamy z funkcji dostępnych w TensorFlow.

Wybór odpowiedniego podziału danych zależy od wielu czynników, takich jak liczba dostępnych danych, złożoność problemu oraz złożoność modelu. Zbyt mały zestaw treningowy może prowadzić do niedouczenia modelu, podczas gdy zbyt mały zestaw walidacyjny lub testowy może prowadzić do niedokładnej oceny wydajności modelu.
    \chapter{Implementacja}
\section{Budowa modelu sieci neuronowej}

W celu rozwiązania problemu rozpoznawania wad produkcyjnych, zastosowano sieć neuronową opartą na architekturze konwolucyjnej (CNN). Sieci tego typu są powszechnie używane w problemach analizy obrazów, ponieważ potrafią efektywnie wykrywać lokalne wzorce i cechy na obrazach. W tym przypadku, sieć będzie w stanie nauczyć się rozpoznawania różnych wad produkcyjnych na podstawie analizy dostarczonych zdjęć.

Model sieci neuronowej został zaimplementowany przy użyciu biblioteki TensorFlow w języku Python. Poniżej przedstawiono kod źródłowy użyty do budowy modelu:

\begin{verbatim}
from tensorflow.keras.models import Sequential
from tensorflow.keras.layers import Conv2D, 
MaxPooling2D, Dense, Flatten, Dropout

model = Sequential()

model.add(Conv2D(16, (3,3), 1, activation='relu', input_shape=(256,256,3)))
model.add(MaxPooling2D())

model.add(Conv2D(32, (3,3), 1, activation='relu'))
model.add(MaxPooling2D())

model.add(Conv2D(16, (3,3), 1, activation='relu'))
model.add(MaxPooling2D())

model.add(Flatten())
model.add(Dense(256, activation='relu'))
model.add(Dense(1, activation='sigmoid'))
\end{verbatim}

Sieć składa się z trzech warstw konwolucyjnych z funkcją aktywacji ReLU oraz warstwami MaxPooling po każdej z nich. Warstwy konwolucyjne uczą się wykrywać lokalne cechy na obrazach, a MaxPooling pomaga w redukcji wymiarowości, co przyspiesza uczenie i zmniejsza ryzyko przeuczenia.

Po trzech warstwach konwolucyjnych, sieć przechodzi przez warstwę Flatten, która spłaszcza dane do jednowymiarowego wektora, co pozwala na przekazanie ich do warstw gęstych (Dense). W tym przypadku, użyto jednej warstwy gęstej z 256 neuronami i funkcją aktywacji ReLU.

Na końcu, sieć kończy się warstwą gęstą z jednym neuronem i funkcją aktywacji sigmoidalną. Ta warstwa odpowiada za generowanie wyników klasyfikacji, gdzie wartość bliska 0 wskazuje na uszkodzony produkt, a wartość bliska 1 na prawidłowy.

Model został skompilowany z użyciem optymalizatora Adam, funkcji straty BinaryCrossentropy oraz metryki dokładności:

\begin{verbatim}
model.compile('adam', loss=tf.losses.BinaryCrossentropy(),
metrics=['accuracy'])
\end{verbatim}

Zastosowanie optymalizatora Adam pomaga w szybszym i skuteczniejszym uczeniu się sieci, gdyż dostosowuje szybkość uczenia na podstawie zmian gradientu. Funkcja straty BinaryCrossentropy jest odpowiednia do problemów klasyfikacji binarnej, takich jak ten, ponieważ pozwala na ocenę różnic między prawdziwymi etykietami a prognozowanymi przez sieć.

\section{Trenowanie modelu}

Trenowanie modelu sieci neuronowej to proces optymalizacji wag w sieci, aby osiągnąć jak najlepszą wydajność w rozpoznawaniu wad produkcyjnych na podstawie zdjęć. W implementacji wykorzystano dane treningowe i walidacyjne do uczenia modelu oraz oceny jego wydajności podczas trenowania. W tej sekcji omówione zostanie trenowanie modelu, analiza wydajności oraz sposób monitorowania procesu uczenia.

Wykorzystano metodę \verb|fit| dostarczoną przez bibliotekę Keras, aby przeszkolić model sieci neuronowej. Poniższy kod przedstawia sposób trenowania modelu przez 5 epok, używając danych treningowych oraz walidacyjnych:

\begin{verbatim}
hist = model.fit(train, epochs=5, 
validation_data=val, callbacks=[tensorboard_callback])
\end{verbatim}

Parametr \verb|epochs| określa liczbę pełnych przebiegów przez dane treningowe. W każdej epoce model próbuje zminimalizować funkcję straty na danych treningowych, dostosowując wagi sieci neuronowej. Użycie danych walidacyjnych pozwala na obserwację procesu uczenia się i wykrycie ewentualnego przeuczenia modelu. Jeśli model zaczyna się przeuczać, dokładność na danych walidacyjnych zacznie maleć, podczas gdy dokładność na danych treningowych nadal będzie rosnąć.

TensorBoard to narzędzie do wizualizacji uczenia sieci neuronowych, które pozwala na monitorowanie różnych metryk, takich jak funkcja straty i dokładność. W implementacji użyto TensorBoard jako wywołania zwrotnego podczas trenowania, co pozwala na automatyczne zapisywanie danych do logów:

\begin{verbatim}
logdir='logs'
tensorboard_callback = tf.keras.callbacks.TensorBoard(log_dir=logdir)
\end{verbatim}

Logi TensorBoard można następnie wyświetlić w przeglądarce, aby uzyskać interaktywną wizualizację procesu uczenia. Aby uruchomić TensorBoard, należy wpisać w terminalu poniższe polecenie, a następnie otworzyć wyświetlony adres URL w przeglądarce:

\begin{verbatim}
tensorboard --logdir=logs
\end{verbatim}

Po zakończeniu trenowania modelu, można ocenić jego wydajność na podstawie historii uczenia. Poniższy kod przedstawia sposób tworzenia wykresów straty oraz dokładności dla danych treningowych i walidacyjnych:

\begin{verbatim}
fig = plt.figure()
plt.plot(hist.history['loss'], color='teal', 
label='strata treningowa')
plt.plot(hist.history['val_loss'], color='orange', 
label='strata walidacji')
fig.suptitle('Strata', fontsize=20)
plt.legend(loc="upper left")
plt.show()

fig = plt.figure()
plt.plot(hist.history['accuracy'], color='teal', 
label='dokładność treningowa')
plt.plot(hist.history['val_accuracy'], color='orange', 
label='dokładność walidacji')
fig.suptitle('Dokładność', fontsize=20)
plt.legend(loc="upper left")
plt.show()
\end{verbatim}

Wykresy te pozwalają na analizę wydajności modelu w czasie trenowania. Na wykresie straty obserwujemy spadek wartości funkcji straty zarówno dla danych treningowych, jak i walidacyjnych. Jeśli model jest odpowiednio uczone, strata na danych walidacyjnych powinna stabilizować się na niskim poziomie.

Wykres dokładności przedstawia, jak dobrze model radzi sobie z klasyfikacją próbek na danych treningowych i walidacyjnych. W miarę jak model się uczy, dokładność powinna rosnąć, aż osiągnie pewien poziom, po którym może wystąpić przeuczenie. W przypadku przeuczenia, dokładność na danych treningowych będzie nadal rosnąć, podczas gdy dokładność na danych walidacyjnych będzie maleć.

\section{Wczytywanie przykładowego obrazu}
W tym oraz w kolejnych podrozdziałach omówione zostanie testowanie modelu na przykładach obrazów, sprawdzając jego zdolność do klasyfikacji zdjęć jako uszkodzonych lub prawidłowych elementów. Testowanie modelu odbywa się na podstawie dostarczonego kodu źródłowego.
Pierwszym krokiem w testowaniu modelu jest wczytanie przykładowego obrazu, który zostanie następnie przekazany do modelu w celu klasyfikacji. Wczytujemy obraz za pomocą biblioteki OpenCV:

\begin{lstlisting}[language=Python]
img = cv2.imread('testing/failure_4.png')
plt.imshow(cv2.cvtColor(img, cv2.COLOR_BGR2RGB))
plt.show()
\end{lstlisting}

\section{Przeskalowanie obrazu}
Przed przekazaniem obrazu do modelu, konieczne jest jego przeskalowanie do wymiarów, na których model został wytrenowany. W naszym przypadku, model został przeszkolony na obrazach o wymiarach 256x256 pikseli. Przeskalowanie obrazu odbywa się za pomocą funkcji \texttt{tf.image.resize}:

\begin{lstlisting}[language=Python]
resize = tf.image.resize(img, (256,256))
plt.imshow(resize.numpy().astype(int))
plt.show()
\end{lstlisting}

\section{Predykcja klasy obrazu}
Następnie przekazujemy przeskalowany obraz do modelu, aby uzyskać predykcję klasy. Przed przekazaniem obrazu do modelu, konieczne jest jego normalizowanie przez podzielenie wartości pikseli przez 255:

\begin{lstlisting}[language=Python]
yhat = model.predict(np.expand_dims(resize/255, 0))
\end{lstlisting}

\section{Interpretacja wyników}
Wynik predykcji (\texttt{yhat}) to wartość od 0 do 1, która wskazuje na przynależność obrazu do klasy 'prawidłowy'. Aby uzyskać konkretną klasę obrazu, ustalamy próg (np. 0,5) i sprawdzamy, czy wartość predykcji przekracza ten próg:

\begin{lstlisting}[language=Python]
if yhat > 0.5:
print(f'Wskazany obraz zostal sklasyfikowany jako czesc prawidlowa')
else:
print(f'Wskazany obraz zostal sklasyfikowany jako czesc uszkodzona')
\end{lstlisting}

W ten sposób możemy przetestować model na różnych przykładach obrazów i ocenić jego zdolność do klasyfikacji uszkodzonych i prawidłowych elementów. Testowanie modelu na przykładach jest istotne dla praktycznego zastosowania modelu w rzeczywistych scenariuszach, gdzie musi on poprawnie klasyfikować różnorodne obrazy przedstawiające uszkodzone i prawidłowe elementy.

\section{Zapisywanie modelu}
Aby zachować wytrenowany model i umożliwić jego dalsze wykorzystanie, należy zapisać jego strukturę i parametry. Dzięki temu będziemy mogli wczytać model i użyć go do klasyfikacji obrazów w przyszłości bez konieczności przeprowadzania procesu treningowego ponownie.
Zapisanie modelu w TensorFlow odbywa się za pomocą metody \texttt{save}:

\begin{lstlisting}[language=Python]
model.save(os.path.join('models', 'imageclassicationversionlive.h5'))
\end{lstlisting}

Model zostaje zapisany w formacie HDF5, który przechowuje zarówno architekturę modelu, jak i nauczone wagi.

\section{Wczytywanie modelu}
Aby wczytać zapisany model, używamy funkcji \texttt{load\_model} z biblioteki TensorFlow:

\begin{lstlisting}[language=Python]
new_model = load_model(os.path.join('models', 'imageclassicationversionlive.h5'))
\end{lstlisting}

Wczytany model można następnie wykorzystać do przewidywania klasy obrazów, podobnie jak przed zapisaniem:

\begin{lstlisting}[language=Python]
img = cv2.imread('testing/failure_4.png')
plt.imshow(cv2.cvtColor(img, cv2.COLOR_BGR2RGB))
plt.show()
resize = tf.image.resize(img, (256,256))
plt.imshow(resize.numpy().astype(int))
plt.show()

yhat = model.predict(np.expand_dims(resize/255, 0))
if yhat > 0.5:
print(f'Wskazany obraz zostal sklasyfikowany jako czesc prawidlowa')
else:
print(f'Wskazany obraz zostal sklasyfikowany jako czesc uszkodzona')
\end{lstlisting}

Dzięki zapisywaniu i wczytywaniu modelu mamy możliwość przechowywania i wykorzystywania wyników treningu w dowolnym momencie, co pozwala na efektywniejsze wykorzystanie zasobów obliczeniowych oraz sprawne wdrażanie modeli do praktycznych zastosowań.
    \chapter{Badania}

\section{Przedmiot badań}
Przedmiotem badań są zdjęcia przedstawiające różne wady produkcyjne oraz model głębokiego uczenia służący do ich rozpoznawania. Zdjęcia zostały podzielone na dwie klasy: prawidłowe i uszkodzone. W ramach badań wykorzystano 20 zdjęć testowych, z których 10 przedstawia prawidłowe części, a pozostałe 10 to zdjęcia uszkodzonych elementów. Eksperymenty przeprowadzono na modelach wyuczonych na różnej liczbie zdjęć, konkretnie na 25, 50, 100, 250 i 500 zdjęciach, aby zbadać wpływ liczby zdjęć użytych do treningu na skuteczność modelu.

\section{Cele badań}
Celem badań jest ocena skuteczności modeli głębokiego uczenia w rozpoznawaniu wad produkcyjnych na zdjęciach, w zależności od liczby zdjęć użytych do wyuczenia modelu. W ramach badań analizowana jest dokładność modeli w rozpoznawaniu zarówno prawidłowych, jak i uszkodzonych części. Badania te mają na celu ocenić, czy model jest w stanie skutecznie klasyfikować zdjęcia w zależności od liczby danych uczących, co może prowadzić do wniosków dotyczących optymalnej liczby zdjęć potrzebnych do skutecznego nauczenia modelu.

\section{Środowisko badawcze}
Do przeprowadzenia badań użyto następujących narzędzi, bibliotek i środowiska programistycznego:
\begin{itemize}
    \item TensorFlow
    \item Keras
    \item OpenCV
    \item Python
\end{itemize}

Tabela 5.1 zawiera szczegółowe informacje na temat konfiguracji sprzętowej i systemu operacyjnego maszyny, na której przeprowadzono badania.

\begin{table}[H]
\centering
\caption{Specyfikacja techniczna maszyny użytej do badań}
\begin{tabular}{|l|l|}
\hline
\textbf{Komponent} & \textbf{Specyfikacja} \\ \hline
Procesor & Intel Core i7-12700K \\ \hline
Pamięć RAM & 32 GB DDR4 \\ \hline
Karta graficzna & Nvidia RTX 3080 Ti \\ \hline
System operacyjny & Fedora 37, Fedora 38 \\ \hline
\end{tabular}
\begin{center}
\footnotesize{Źródło: opracowanie własne}
\end{center}
\end{table}

\section{Metoda badawcza}
W celu oceny skuteczności modeli zastosowano następujące metody:
\begin{itemize}
    \item \textbf{Analiza straty i dokładności}: porównanie wartości straty (błędu) i dokładności (poprawnych klasyfikacji) dla danych treningowych i walidacyjnych w trakcie procesu uczenia. Wartości te są zapisywane podczas każdej epoki treningowej, co pozwala na obserwację, jak model się uczy, i może pomóc w identyfikacji problemów, takich jak przetrenowanie.
    
    \item \textbf{Analiza przypadków sukcesów i błędów}: szczegółowa analiza poszczególnych przypadków, w których model dokonał poprawnej lub błędnej klasyfikacji. Przeglądanie tych przypadków może pomóc w zrozumieniu, jakie cechy zdjęć wpływają na decyzje modelu i gdzie model może się mylić.
\end{itemize}

\section{Trenowanie modelu}
Trenowanie modelu polega na uczeniu sieci neuronowej na podstawie dostarczonego zbioru danych uczących. W trakcie procesu trenowania, model optymalizuje swoje wagi, aby zminimalizować stratę wynikającą z różnicy między prognozami modelu a rzeczywistymi etykietami danych uczących.

Fragment kodu odpowiadający za trenowanie modelu:

\begin{verbatim}
    hist = model.fit(train, epochs=15, validation_data=val, 
    callbacks=[tensorboard_callback])
\end{verbatim}

\section{Walidacja}
Walidacja to proces oceny wydajności modelu na podzbiorze danych, który nie był używany do trenowania modelu. Walidacja pozwala na monitorowanie postępów uczenia się modelu, a także na wykrywanie sytuacji, w których model jest przetrenowany. W trakcie walidacji model nie jest aktualizowany, a jedynie oceniany na podstawie danych walidacyjnych.

\section{Testowanie}
Testowanie polega na ocenie wydajności nauczonego modelu na zupełnie nowym, nieznanym zbiorze danych (zbiór testowy). Proces ten pozwala na rzeczywistą ocenę, jak dobrze model radzi sobie z prognozowaniem na danych, które nie były używane ani do trenowania, ani do walidacji. Wyniki testowania mogą dać szacunkową informację na temat tego, jak model poradzi sobie z danymi napotkanymi w rzeczywistych zastosowaniach.

Fragment kodu odpowiadający za testowanie modelu:

\begin{verbatim}
    img = cv2.imread('testing/failure_10.png')
    plt.imshow(cv2.cvtColor(img, cv2.COLOR_BGR2RGB))
    plt.show()

    resize = tf.image.resize(img, (256,256))
    plt.imshow(resize.numpy().astype(int))
    plt.show()

    yhat  = model.predict(np.expand_dims(resize/255, 0))
    if yhat > 0.5:
    print(f'Wskazany obraz został sklasyfikowany jako część prawidłowa')
    else:
    print(f'Wskazany obraz został sklasyfikowany jako część uszkodzona')
\end{verbatim}

\section{Materiały badawcze}

W niniejszym podrozdziale przedstawione zostaną materiały badawcze wykorzystane do przeprowadzenia eksperymentów. Wykorzystano 20 zdjęć testowych podzielonych na dwie klasy: prawidłowe i uszkodzone. Każda z klas zawiera 10 zdjęć. Materiały te zostały wygenerowane przy użyciu sztucznej inteligencji na wzór zdjęć z rzeczywistego procesu produkcyjnego.

Zdjęcia te symulują różne sytuacje, które mogą wystąpić w rzeczywistości, takie jak różne rodzaje wad, różnorodność kształtów i rozmiarów części oraz różne warunki oświetleniowe. Wykorzystanie takich zdjęć do przeprowadzenia badań pozwala na ocenę skuteczności modeli głębokiego uczenia w warunkach zbliżonych do rzeczywistych, co może prowadzić do bardziej wiarygodnych wyników i przydatnych wniosków.

Wykorzystanie zdjęć wygenerowanych przez sztuczną inteligencję pozwala również na kontrolowanie liczby danych uczących oraz dokładne dopasowanie stopnia trudności zdjęć. W ten sposób można zbadać wpływ liczby zdjęć użytych do treningu na skuteczność modelu oraz zrozumieć, jakie cechy zdjęć wpływają na decyzje modelu i gdzie model może się mylić.

W pracy przedstawiono zbiór zdjęć produktów użytych do testowania. Zdjęcia podzielone są na dwie kategorie: prawidłowe produkty oraz uszkodzone produkty. 

Na zdjęciach prawidłowych produktów (Rys.~\ref{fig:zdjecie_poprawne_1}--\ref{fig:zdjecie_poprawne_10}) prezentowane są różne typy produktów spełniających wymagane kryteria jakości. Ich celem jest prezentacja dobrych praktyk produkcyjnych oraz jako punkt odniesienia dla porównania z uszkodzonymi produktami.

Natomiast na zdjęciach uszkodzonych produktów (Rys.~\ref{fig:zdjecie_uszkodzone_1}--\ref{fig:zdjecie_uszkodzone_10}) przedstawiono przykłady wadliwych produktów, które nie spełniają wymagań jakościowych. Mogą one zawierać pęknięcia, złamania, nierówności, deformacje czy inne wady, które wpływają na ich funkcjonalność i estetykę.

\begin{figure}[htbp]
  \centering
  \caption{Prawidłowe - zdjęcie 1}
  \includegraphics[width=150px]{images/success_1.png}
  \begin{center}
  \footnotesize{Źródło: opracowanie własne}
  \end{center}
  \label{fig:zdjecie_poprawne_1}
\end{figure}

\begin{figure}[htbp]
  \centering
  \caption{Prawidłowe - zdjęcie 2}
  \includegraphics[width=150px]{images/success_2.png}
  \begin{center}
  \footnotesize{Źródło: opracowanie własne}
  \end{center}
  \label{fig:zdjecie_poprawne_2}
\end{figure}

\begin{figure}[htbp]
  \centering
  \caption{Prawidłowe - zdjęcie 3}
  \includegraphics[width=150px]{images/success_3.png}
  \begin{center}
  \footnotesize{Źródło: opracowanie własne}
  \end{center}
  \label{fig:zdjecie_poprawne_1}
\end{figure}

\begin{figure}[htbp]
  \centering
  \caption{Prawidłowe - zdjęcie 4}
  \includegraphics[width=150px]{images/success_4.png}
  \begin{center}
  \footnotesize{Źródło: opracowanie własne}
  \end{center}
  \label{fig:zdjecie_poprawne_4}
\end{figure}

\begin{figure}[htbp]
  \centering
  \caption{Prawidłowe - zdjęcie 5}
  \includegraphics[width=150px]{images/success_5.png}
  \begin{center}
  \footnotesize{Źródło: opracowanie własne}
  \end{center}
  \label{fig:zdjecie_poprawne_5}
\end{figure}

\begin{figure}[htbp]
  \centering
  \caption{Prawidłowe - zdjęcie 6}
  \includegraphics[width=150px]{images/success_6.png}
  \begin{center}
  \footnotesize{Źródło: opracowanie własne}
  \end{center}
  \label{fig:zdjecie_poprawne_6}
\end{figure}

\begin{figure}[htbp]
  \centering
  \caption{Prawidłowe - zdjęcie 7}
  \includegraphics[width=150px]{images/success_7.png}
  \begin{center}
  \footnotesize{Źródło: opracowanie własne}
  \end{center}
  \label{fig:zdjecie_poprawne_1}
\end{figure}

\begin{figure}[htbp]
  \centering
  \caption{Prawidłowe - zdjęcie 8}
  \includegraphics[width=150px]{images/success_8.png}
  \begin{center}
  \footnotesize{Źródło: opracowanie własne}
  \end{center}
  \label{fig:zdjecie_poprawne_8}
\end{figure}

\begin{figure}[htbp]
  \centering
  \caption{Prawidłowe - zdjęcie 9}
  \includegraphics[width=150px]{images/success_9.png}
  \begin{center}
  \footnotesize{Źródło: opracowanie własne}
  \end{center}
  \label{fig:zdjecie_poprawne_9}
\end{figure}

\begin{figure}[htbp]
  \centering
  \caption{Prawidłowe - zdjęcie 10}
  \includegraphics[width=150px]{images/success_10.png}
  \begin{center}
  \footnotesize{Źródło: opracowanie własne}
  \end{center}
  \label{fig:zdjecie_poprawne_10}
\end{figure}

\begin{figure}[htbp]
  \centering
  \caption{Uszkodzone - zdjęcie 1}
  \includegraphics[width=150px]{images/failure_1.png}
  \begin{center}
  \footnotesize{Źródło: opracowanie własne}
  \end{center}
  \label{fig:zdjecie_uszkodzone_1}
\end{figure}

\begin{figure}[htbp]
  \centering
  \caption{Uszkodzone - zdjęcie 2}
  \includegraphics[width=150px]{images/failure_2.png}
  \begin{center}
  \footnotesize{Źródło: opracowanie własne}
  \end{center}
  \label{fig:zdjecie_uszkodzone_2}
\end{figure}

\begin{figure}[htbp]
  \centering
  \caption{Uszkodzone - zdjęcie 3}
  \includegraphics[width=150px]{images/failure_3.png}
  \begin{center}
  \footnotesize{Źródło: opracowanie własne}
  \end{center}
  \label{fig:zdjecie_uszkodzone_3}
\end{figure}

\begin{figure}[htbp]
  \centering
  \caption{Uszkodzone - zdjęcie 4}
  \includegraphics[width=150px]{images/failure_4.png}
  \begin{center}
  \footnotesize{Źródło: opracowanie własne}
  \end{center}
  \label{fig:zdjecie_uszkodzone_4}
\end{figure}

\begin{figure}[htbp]
  \centering
  \caption{Uszkodzone - zdjęcie 5}
  \includegraphics[width=150px]{images/failure_5.png}
  \begin{center}
  \footnotesize{Źródło: opracowanie własne}
  \end{center}
  \label{fig:zdjecie_uszkodzone_5}
\end{figure}

\begin{figure}[htbp]
  \centering
  \caption{Uszkodzone - zdjęcie 6}
  \includegraphics[width=150px]{images/failure_6.png}
  \begin{center}
  \footnotesize{Źródło: opracowanie własne}
  \end{center}
  \label{fig:zdjecie_uszkodzone_6}
\end{figure}

\begin{figure}[htbp]
  \centering
  \caption{Uszkodzone - zdjęcie 7}
  \includegraphics[width=150px]{images/failure_7.png}
  \begin{center}
  \footnotesize{Źródło: opracowanie własne}
  \end{center}
  \label{fig:zdjecie_uszkodzone_7}
\end{figure}

\begin{figure}[htbp]
  \centering
  \caption{Uszkodzone - zdjęcie 8}
  \includegraphics[width=150px]{images/failure_8.png}
  \begin{center}
  \footnotesize{Źródło: opracowanie własne}
  \end{center}
  \label{fig:zdjecie_uszkodzone_8}
\end{figure}

\begin{figure}[htbp]
  \centering
  \caption{Uszkodzone - zdjęcie 9}
  \includegraphics[width=150px]{images/failure_9.png}
  \begin{center}
  \footnotesize{Źródło: opracowanie własne}
  \end{center}
  \label{fig:zdjecie_uszkodzone_9}
\end{figure}

\begin{figure}[htbp]
  \centering
  \caption{Uszkodzone - zdjęcie 10}
  \includegraphics[width=150px]{images/failure_10.png}
  \begin{center}
  \footnotesize{Źródło: opracowanie własne}
  \end{center}
  \label{fig:zdjecie_uszkodzone_10}
\end{figure}
    \chapter{Prezentacja działania programu}
    \chapter{Podsumowanie}
\section{Wnioski}
W wyniku przeprowadzonych badań można wysnuć następujące wnioski:

\begin{enumerate}
\item Liczba zdjęć użytych do wyuczenia modelu ma znaczący wpływ na jego dokładność w rozpoznawaniu wad produkcyjnych. Wraz ze wzrostem liczby zdjęć użytych do nauki, dokładność modelu zwykle się poprawia.
\item Model wyuczony na 500 zdjęciach osiągnął najwyższą dokładność w testach, co sugeruje, że większa liczba danych uczących przyczynia się do lepszej generalizacji modelu.
\item W przypadku małych zbiorów danych, model może być narażony na przetrenowanie, co skutkuje słabszymi wynikami na danych testowych.
\item Potrzeba dalszych badań nad optymalizacją architektury modelu oraz eksploracją innych technik przetwarzania wstępnego danych, takich jak augmentacja danych, aby poprawić wydajność modelu.
\item Zastosowanie różnych technik uczenia transferowego może przyczynić się do zwiększenia skuteczności modelu, szczególnie w przypadku mniejszych zbiorów danych.
\end{enumerate}

W oparciu o przeprowadzone badania, można zidentyfikować następujące kierunki dalszego rozwoju projektu:

\begin{itemize}
\item Eksploracja innych architektur sieci głębokiego uczenia, takich jak sieci ResNet, DenseNet lub Inception, które mogą osiągać lepsze wyniki w rozpoznawaniu wad produkcyjnych.
\item Zastosowanie technik augmentacji danych, aby zwiększyć liczbę dostępnych danych uczących oraz poprawić generalizację modelu.
\item Implementacja mechanizmów regularyzacji, takich jak dropout, aby zmniejszyć ryzyko przetrenowania modelu, zwłaszcza w przypadku małych zbiorów danych.
\item Analiza wpływu różnych parametrów uczenia, takich jak rozmiar wsadu, współczynnik uczenia czy optymalizator, na dokładność modelu.
\item Badanie zastosowania uczenia transferowego, wykorzystując wytrenowane na dużych zbiorach danych modele do ekstrakcji cech, co może skrócić czas uczenia oraz poprawić skuteczność modelu na mniejszych zbiorach danych.
\item Eksploracja zastosowań modelu w innych dziedzinach, takich jak analiza jakości innych produktów, gdzie istnieje potrzeba wykrywania wad na podstawie zdjęć.
\end{itemize}

W kontekście praktycznym, wyniki badań oraz analizy mogą być wykorzystane do planowania i implementacji systemów kontroli jakości w różnych gałęziach przemysłu. Poprawa wydajności modelu w wykrywaniu wad produkcyjnych może przyczynić się do redukcji kosztów związanych z wadliwymi produktami oraz poprawy ogólnej jakości oferowanych produktów. W dłuższej perspektywie, opracowanie efektywnego i dokładnego modelu rozpoznawania wad może także pomóc w automatyzacji procesów kontroli jakości, co pozwoli na optymalizację zasobów i zwiększenie efektywności produkcji.

\section{Zrealizowane cele}

W ramach niniejszej pracy pt. "Rozpoznawanie wad produkcyjnych z wykorzystaniem uczenia głębokiego" osiągnięto szereg celów, które przyczyniły się do zrozumienia i wykorzystania technik uczenia głębokiego w celu wykrywania wad produkcyjnych. Poniżej przedstawiamy szczegółowy opis zrealizowanych celów:

\begin{itemize}
\item \textbf{Przegląd literatury :} Przeprowadzono obszerny przegląd literatury, który obejmował analizę podstawowych koncepcji uczenia głębokiego, sieci neuronowych, konwolucyjnych sieci neuronowych (CNN) oraz przegląd zastosowań tych technik w różnych dziedzinach przemysłu. Analiza literatury pozwoliła na zrozumienie podstawowych zagadnień związanych z uczeniem głębokim oraz na identyfikację kluczowych technik i metod, które mogą być wykorzystane w celu rozpoznawania wad produkcyjnych.
\item \textbf{Przygotowanie danych uczących}
Zebrano i przygotowano zbiór danych uczących, który obejmował zdjęcia przedstawiające produkty z wadami oraz produkty prawidłowe. Przeprowadzono wstępne przetwarzanie danych, takie jak skalowanie, kadrowanie i normalizację, aby ułatwić proces uczenia modelu. Przygotowanie odpowiedniego zbioru danych uczących stanowi podstawę dla późniejszego etapu uczenia modelu oraz oceny jego skuteczności.
\item \textbf{Budowa i uczenie modeli}
Zbudowano modele konwolucyjnych sieci neuronowych (CNN) z różnymi parametrami, takimi jak liczba warstw, rozmiar filtrów, funkcje aktywacji czy optymalizatory. Przeprowadzono uczenie modeli na różnych zbiorach danych uczących, obejmujących 25, 50, 100 oraz 500 zdjęć, co pozwoliło na analizę wpływu liczby danych uczących na skuteczność modelu.
\item \textbf{Ocena i analiza wyników}
Przeprowadzono testy modeli na danych testowych, które nie były wykorzystywane podczas uczenia. Analizowano skuteczność modeli na podstawie ich zdolności do poprawnego rozpoznawania produktów z wadami i produktów prawidłowych. Wyniki testów posłużyły do oceny ogólnej skuteczności modeli oraz do identyfikacji obszarów wymagających dalszych usprawnień.
\item \textbf{Wnioski i kierunki dalszego rozwoju}
Na podstawie przeprowadzonych badań sformułowano wnioski dotyczące zastosowania uczenia głębokiego w rozpoznawaniu wad produkcyjnych oraz zidentyfikowano kierunki dalszego rozwoju projektu. Wskazano między innymi, że liczba zdjęć użytych do wyuczenia modelu ma znaczący wpływ na jego dokładność, a model wyuczony na większej liczbie zdjęć osiąga lepsze wyniki. Zauważono również, że dalsze badania nad optymalizacją architektury modelu oraz eksploracją innych technik przetwarzania wstępnego danych mogą przyczynić się do poprawy wydajności modelu.
W ramach dalszego rozwoju projektu zidentyfikowano następujące kierunki:
\begin{itemize}
\item Eksploracja innych architektur sieci głębokiego uczenia, takich jak sieci ResNet, DenseNet lub Inception, które mogą osiągać lepsze wyniki w rozpoznawaniu wad produkcyjnych.
\item Zastosowanie technik augmentacji danych, aby zwiększyć liczbę dostępnych danych uczących oraz poprawić generalizację modelu.
\item Implementacja mechanizmów regularyzacji, takich jak dropout, aby zmniejszyć ryzyko przetrenowania modelu, zwłaszcza w przypadku małych zbiorów danych.
\item Analiza wpływu różnych parametrów uczenia, takich jak rozmiar wsadu, współczynnik uczenia czy optymalizator, na dokładność modelu.
\end{itemize}

\end{itemize}
\endgroup
\medskip

\renewcommand\bibname{\vspace{-2cm}\large{Bibliografia}}
\renewcommand\listfigurename{\vspace{-2cm}\large{Spis rysunków}}
\renewcommand\listtablename{\vspace{-2cm}\large{Spis tabel}}
\begin{thebibliography}{12} 
% \vspace{-1cm}
\bibitem{book} Yoshua Bengio, Ian Goodfellow, Aaron Courville, Deep Learning Systemy uczące się, Wydawnictwo Naukowe PWN, Warszawa, 2018.
\bibitem{book} Mariusz Flasiński, Wstęp do sztucznej inteligencji, Wydawnictwo Naukowe PWN, Warszawa, 2018.
\bibitem{book} Joel Grus, Data science od podstaw. Analiza danych w Pythonie, Helion, Gliwice, 2022.
\bibitem{book} Ryszard Knosala, Inżynieria produkcji, PWE Polskie Wydawnictwo Ekonomiczne, Warszawa 2017.
\bibitem{book} Robert A. Kosiński, Sztuczne sieci neuronowe, Wydawnictwo Naukowe PWN, Warszawa, 2017.
\bibitem{book} Yuxi Hayden Liu, Python Uczenie maszynowe w przykładach TensorFlow 2 PyTorch i scikitlearn, Helion, Gliwice, 2022.
\bibitem{book} Mark Lutz, Python. Wprowadzenie, Helion, Gliwice, 2022.
\bibitem{book} Sebastian Raschka, Vahid Mirjalili, Python Machine learning i deep learning, Helion, Gliwice, 2021.
\bibitem{book} Kazimierz Szatkowski, Nowoczesne zarządzanie produkcją, Wydawnictwo Naukowe PWN, Warszawa 2023.
\bibitem{book} Jerzy Surma, Hakowanie sztucznej inteligencji, Wydawnictwo Naukowe PWN, Warszawa, 2022.
\end{thebibliography}

\newpage
\listoffigures
\newpage
\listoftables

\newpage

\chapter*{}
\centering\textbf{OŚWIADCZENIE\\[6mm]}
\raggedright
Oświadczam, że: 
\begin{enumerate}
    \item pracę niniejszą przygotowałem(am) samodzielnie; wszystkie dane, istotne myśli i sformułowania pochodzące 
z literatury (przytoczone dosłownie lub niedosłownie) są opatrzone odpowiednimi odsyłaczami; praca ta nie była 
w całości ani w części przez nikogo przedkładana do żadnej oceny i nie była publikowana;
    \item wyrażam zgodę / nie wyrażam zgody na udostępnianie mojej pracy dyplomowej. 
\end{enumerate}\\[10mm]
Data ..............................\\[15mm]


\centering
\hspace{200pt} ..............................................\\[0mm]
\hspace{200pt} imię i nazwisko\\[10mm]
\hspace{200pt} Stwierdzam autentyczność podpisu\\[5mm]
\hspace{200pt} .....................................................................\\[0mm]
\hspace{200pt} (podpis  pracownika i pieczątka Wydziału)\\[20mm]
 
 \raggedright
 * niepotrzebne skreślić
\end{document}