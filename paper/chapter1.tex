\chapter{Wstęp}
\section{Cel i motywacja pracy}

Celem niniejszej pracy magisterskiej jest wykorzystanie i zastosowanie metod uczenia głębokiego w celu rozpoznawania wad produkcyjnych na podstawie analizy zdjęć. Ze względu na coraz większe zapotrzebowanie na szybkie i efektywne rozwiązania w zakresie kontroli jakości, zastosowanie technik sztucznej inteligencji staje się niezbędne dla przemysłu. W szczególności, wykorzystanie uczenia głębokiego, jako jednego z najbardziej zaawansowanych podejść w dziedzinie sztucznej inteligencji, pozwala na znaczące zwiększenie skuteczności wykrywania wad produkcyjnych.

Motywacją do podjęcia tego tematu była chęć eksploracji i zrozumienia nowoczesnych technologii związanych z uczeniem maszynowym oraz ich praktyczne zastosowanie w kontekście przemysłowym. Wadliwe komponenty mogą prowadzić do znacznych strat finansowych oraz wpływać negatywnie na reputację przedsiębiorstwa. Automatyczne wykrywanie wad na wczesnym etapie procesu produkcyjnego może przyczynić się do zwiększenia efektywności, zmniejszenia kosztów oraz ograniczenia ilości produktów o niskiej jakości trafiających do odbiorców.

W ramach pracy magisterskiej opracowany zostanie system, który będzie w stanie analizować zdjęcia przedmiotów i automatycznie klasyfikować je jako wadliwe lub prawidłowe. System ten będzie oparty na technikach uczenia głębokiego, takich jak konwolucyjne sieci neuronowe, które są obecnie szeroko stosowane w różnych dziedzinach analizy obrazów. Praca będzie obejmować zarówno teorię, jak i praktyczne aspekty projektowania oraz implementacji.

W celu zilustrowania koncepcji i metod przedstawionych w pracy, zostanie przeprowadzone eksperymentalne zastosowanie opracowanego systemu do wybranego zbioru zdjęć reprezentujących obiekty z wadami produkcyjnymi oraz prawidłowymi komponentami. Wyniki tego eksperymentu posłużą jako dowód na skuteczność zastosowanego podejścia oraz jako punkt wyjścia do dalszej dyskusji na temat potencjalnych ulepszeń i przyszłych kierunków rozwoju.

\section{Zawartość pracy}

W niniejszej pracy przedstawiono zagadnienia związane z rozpoznawaniem wad produkcyjnych z wykorzystaniem uczenia głębokiego. Poniżej przedstawiono opis zawartości pracy według kolejnych rozdziałów:

\begin{itemize}
\item \textbf{Rozdział 1} - Wstęp: W tym rozdziale omówiono cel i motywację pracy oraz przedstawiono zawartość pracy.
\item \textbf{Rozdział 2} - Wybrane podstawy teoretyczne pracy: W drugim rozdziale omówiono podstawy teoretyczne pracy, takie jak uczenie głębokie, sieci konwolucyjne, rodzaje wad produkcyjnych oraz istniejące rozwiązania związane z kontrolą jakości. Przedstawiono także język programowania wykorzystany w projekcie.
\item \textbf{Rozdział 3} - Projekt systemu: W tym rozdziale przedstawiono wymagania funkcjonalne i niefunkcjonalne oraz ogólny zarys rozwiązania. Opisano także wybór sieci konwolucyjnej oraz korzyści z zastosowania uczenia głębokiego.
\item \textbf{Rozdział 4} - Implementacja: W czwartym rozdziale omówiono budowę modelu sieci neuronowej oraz proces trenowania i testowania modelu na przykładach. Przedstawiono także sposób zapisywania i wczytywania modelu.
\item \textbf{Rozdział 5} - Badania: W piątym rozdziale opisano przedmiot, cele oraz metodę badań. Przedstawiono również środowisko badawcze oraz przebieg badań.
\item \textbf{Rozdział 6} - Wyniki i analiza badań: W tym rozdziale przedstawiono wyniki dla modeli uczenia głębokiego wyuczonych na różnych liczbach zdjęć oraz przeprowadzono analizę tych wyników.
\item \textbf{Rozdział 7} - Podsumowanie: W ostatnim rozdziale podsumowano pracę, przedstawiono wnioski oraz opisano zrealizowane cele. Wskazano także kierunki dalszego rozwoju projektu.
\end{itemize}