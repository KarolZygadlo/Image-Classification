\chapter{Wstęp}
\section{Cel i motywacja pracy}

Celem niniejszej pracy magisterskiej jest opracowanie i zastosowanie metod uczenia głębokiego w celu rozpoznawania wad produkcyjnych na podstawie analizy zdjęć. Ze względu na coraz większe zapotrzebowanie na szybkie i efektywne rozwiązania w zakresie kontroli jakości, zastosowanie technik sztucznej inteligencji staje się niezbędne dla przemysłu. W szczególności, wykorzystanie uczenia głębokiego, jako jednego z najbardziej zaawansowanych podejść w dziedzinie sztucznej inteligencji, pozwala na znaczące zwiększenie skuteczności wykrywania wad produkcyjnych.

Motywacją do podjęcia tego tematu była chęć eksploracji i zrozumienia nowoczesnych technologii związanych z uczeniem maszynowym oraz ich praktyczne zastosowanie w kontekście przemysłowym. Wadliwe komponenty mogą prowadzić do znacznych strat finansowych oraz wpływać negatywnie na reputację przedsiębiorstwa. Automatyczne wykrywanie wad na wczesnym etapie procesu produkcyjnego może przyczynić się do zwiększenia efektywności, zmniejszenia kosztów oraz ograniczenia ilości produktów o niskiej jakości trafiających do odbiorców.

W ramach pracy magisterskiej opracowany zostanie system, który będzie w stanie analizować zdjęcia przedmiotów i automatycznie klasyfikować je jako wadliwe lub prawidłowe. System ten będzie oparty na technikach uczenia głębokiego, takich jak konwolucyjne sieci neuronowe, które są obecnie szeroko stosowane w różnych dziedzinach analizy obrazów. Praca będzie obejmować zarówno teorię, jak i praktyczne aspekty projektowania, implementacji oraz oceny tego rodzaju systemów.

W celu zilustrowania koncepcji i metod przedstawionych w pracy, zostanie przeprowadzone eksperymentalne zastosowanie opracowanego systemu do wybranego zbioru zdjęć reprezentujących obiekty z wadami produkcyjnymi oraz prawidłowymi komponentami. Wyniki tego eksperymentu posłużą jako dowód na skuteczność zastosowanego podejścia oraz jako punkt wyjścia do dalszej dyskusji na temat potencjalnych ulepszeń i przyszłych kierunków rozwoju.

\section{Zawartość pracy}

W niniejszej pracy dyplomowej skupiamy się na zagadnieniu wykrywania wad produkcyjnych z wykorzystaniem uczenia głębokiego, w szczególności sieci konwolucyjnych. Zakres pracy obejmuje następujące aspekty:

\begin{itemize}
\item Zaprezentowanie wybranych podstaw teoretycznych związanych z uczeniem głębokim, sieciami konwolucyjnymi oraz ich zastosowaniem w przemyśle;
\item Analiza istniejących rozwiązań i technologii stosowanych w kontroli jakości, w tym systemów wizyjnych, Internetu Rzeczy (IoT) oraz analizy danych;
\item Opracowanie projektu systemu, który spełnia określone wymagania funkcjonalne i niefunkcjonalne, oraz omówienie kroków projektowania takich jak wybór sieci konwolucyjnej, wczytywanie danych czy wstępne przetwarzanie danych;
\item Implementacja algorytmu uczenia głębokiego na podstawie opracowanego projektu, przedstawienie budowy modelu sieci neuronowej oraz etapów trenowania i testowania modelu;
\item Ewaluacja i analiza wyników uzyskanych podczas testów skuteczności i wydajności opracowanego systemu;
\item Prezentacja działania programu na przykładach zastosowań;
\item Podsumowanie wyników pracy, wskazanie zrealizowanych celów oraz przedstawienie perspektyw na dalsze badania i rozwój systemu.
\end{itemize}

W ramach pracy został opracowany kod źródłowy, który prezentuje implementację modelu sieci neuronowej oraz wszystkie etapy przetwarzania danych, trenowania, testowania i ewaluacji modelu. Kod źródłowy został napisany w wybranym języku programowania i korzysta z odpowiednich bibliotek oraz narzędzi dedykowanych dla uczenia głębokiego i sieci konwolucyjnych.