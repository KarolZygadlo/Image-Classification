\chapter{Wymagania funkcjonalne oraz niefunkcjonalne}

\section{Wymagania funkcjonalne}
Wymagania funkcjonalne systemu obejmują następujące elementy:

\begin{enumerate}
\item \textbf{Wczytywanie i przetwarzanie danych wejściowych (obrazów):} System powinien być w stanie wczytać dane wejściowe w postaci obrazów oraz przetworzyć je w celu przygotowania do analizy przez model.

\item \textbf{Trenowanie modelu na zbiorze treningowym:} System powinien być wyposażony w model oparty na uczeniu głębokim, który jest trenowany na zbiorze treningowym, zawierającym obrazy uszkodzonych i prawidłowych elementów.

\item \textbf{Walidacja modelu na zbiorze walidacyjnym:} System powinien wykorzystywać zbiór walidacyjny, aby sprawdzić jakość modelu w trakcie procesu trenowania. Walidacja pozwala dostosować hiperparametry modelu, aby uniknąć nadmiernego dopasowania (overfitting).

\item \textbf{Testowanie modelu na zbiorze testowym:} Po zakończeniu trenowania, system powinien zostać przetestowany na zbiorze testowym, który zawiera obrazy nieznane dla modelu. Wyniki testów pozwolą ocenić ostateczną jakość modelu.

\item \textbf{Klasyfikacja obrazów na części uszkodzone i prawidłowe:} Głównym celem systemu jest klasyfikacja obrazów na części uszkodzone i prawidłowe, co pozwala zidentyfikować problemy z jakością w procesie produkcyjnym.

\end{enumerate}

\section{Wymagania niefunkcjonalne}
Wymagania niefunkcjonalne systemu odnoszą się do cech jakościowych, takich jak:

\begin{enumerate}

\item \textbf{Dokładność klasyfikacji:} System powinien osiągać wysoką dokładność klasyfikacji, aby skutecznie identyfikować uszkodzone i prawidłowe elementy.

\item \textbf{Czas uczenia modelu:} Czas trenowania modelu powinien być na tyle krótki, aby umożliwić szybkie dostosowanie modelu do nowych danych.

\item \textbf{Złożoność obliczeniowa modelu:} Model powinien być na tyle prosty, aby możliwe obliczenia nie obciążały nadmiernie zasobów sprzętowych, jednocześnie zachowując wysoką jakość klasyfikacji.

\item \textbf{Skalowalność systemu:} System powinien być skalowalny, co oznacza, że powinien być w stanie obsłużyć większe ilości danych oraz dostosować się do zmieniających się warunków (np. dodanie nowych klas obiektów do klasyfikacji).

\item \textbf{Współczynnik fałszywych pozytywów i fałszywych negatywów:} System powinien charakteryzować się niskim współczynnikiem fałszywych pozytywów (FP) i fałszywych negatywów (FN). Fałszywe pozytywy to przypadki, gdy system błędnie klasyfikuje uszkodzone elementy jako prawidłowe, natomiast fałszywe negatywy to błędna klasyfikacja prawidłowych elementów jako uszkodzone. Oba te rodzaje błędów mogą prowadzić do niekorzystnych skutków, takich jak przestój w produkcji, czy też przekroczenie progów jakościowych.

\end{enumerate}

\chapter{Projekt rozwiązania}

\section{Wybór sieci konwolucyjnej}
Sieci konwolucyjne są szczególnie odpowiednie dla problemów związanych z analizą obrazów, ponieważ potrafią automatycznie uczyć się cech na różnych poziomach abstrakcji. W przeciwnym razie, ręczne projektowanie cech obrazu, zwłaszcza w przypadku złożonych zadań klasyfikacji, może być żmudne i czasochłonne. CNN pozwala nam wykryć lokalne wzorce w obrazach, takie jak kształty i tekstury, które są istotne dla zadania klasyfikacji.

\section{Korzyści z zastosowania uczenia głębokiego}
Uczenie głębokie, jako poddziedzina uczenia maszynowego, oferuje wiele korzyści w kontekście klasyfikacji obrazów. Oto niektóre z nich:

\begin{itemize}
\item \textbf{Automatyczna ekstrakcja cech:} Uczenie głębokie pozwala na automatyczne wykrywanie istotnych cech w danych, eliminując potrzebę ręcznego projektowania cech. W przypadku klasyfikacji obrazów oznacza to, że sieci głębokie potrafią uczyć się hierarchicznych reprezentacji obrazów, co prowadzi do lepszej wydajności klasyfikacji.

\item \textbf{Generalizacja:} Uczenie głębokie ma zdolność do generalizacji na nowe, niewidziane wcześniej dane. Oznacza to, że model wytrenowany na odpowiednio dużym i różnorodnym zbiorze danych może skutecznie klasyfikować obrazy, które nie były częścią jego zbioru treningowego.

\item \textbf{Skalowalność:} Architektury uczenia głębokiego są elastyczne i łatwo skalowalne. Można je dostosować do różnych rozmiarów i rodzajów danych, co pozwala na efektywne rozwiązanie problemów o różnym stopniu złożoności.

\item \textbf{Wydajność:} Dzięki zastosowaniu akceleratorów sprzętowych, takich jak GPU, proces uczenia głębokich sieci neuronowych można znacznie przyspieszyć, co prowadzi do szybszego rozwoju i wdrożenia modeli.

\end{itemize}

\section{Ogólny zarys rozwiązania}
Zaprogramowane rozwiązanie będzie oparte na sieci konwolucyjnej (CNN), która będzie trenowana na danych obrazowych przedstawiających uszkodzone i prawidłowe elementy. Przygotowany model będzie następnie testowany na zbiorze testowym w celu oceny jego dokładności oraz zdolności generalizacji.

W kolejnych podrozdziałach przedstawione zostną szczegółowe etapy procesu projektowania rozwiązania, takie jak:

\begin{itemize}
\item Import bibliotek
\item Wczytywanie danych
\item Ładowanie danych
\item Wstępne przetwarzanie danych
\item Podział danych na zestawy treningowe, walidacyjne i testowe
\end{itemize}

Na podstawie uzyskanych wyników będzie można ocenić jakość opracowanego rozwiązania oraz jego potencjalne ograniczenia. 

\section{Import bibliotek}
W tym podrozdziale omówione zostaną wykorzystane biblioteki oraz ich zastosowanie w projekcie. W naszym rozwiązaniu korzystamy z następujących bibliotek:

\begin{itemize}
\item \textbf{TensorFlow:} Główna biblioteka do uczenia głębokiego, która umożliwia definiowanie, trenowanie i ewaluację modeli sieci neuronowych. W projekcie wykorzystany jest TensorFlow do implementacji sieci konwolucyjnej (CNN) oraz zarządzania procesem uczenia i walidacji modelu.

\item \textbf{OpenCV:} Popularna biblioteka do przetwarzania obrazów, która umożliwia wczytywanie, modyfikowanie i zapisywanie obrazów w różnych formatach. W projekcie wykorzystany jest OpenCV do wczytywania obrazów z dysku, ich przekształceń (np. skalowanie, obrót), a także sprawdzania poprawności danych obrazowych.

\item \textbf{Matplotlib:} Biblioteka do generowania wysokiej jakości wykresów 2D i 3D. W projekcie wykorzystany jest Matplotlib do wizualizacji danych, takich jak wykresy dokładności i straty w trakcie procesu uczenia oraz prezentacji wyników klasyfikacji obrazów.

\item \textbf{NumPy:} Biblioteka do obsługi macierzy wielowymiarowych, która oferuje wiele funkcji matematycznych o dużej wydajności. W projekcie wykorzystany jest NumPy do manipulacji danymi obrazowymi, przetwarzania macierzy oraz realizacji obliczeń matematycznych.

\end{itemize}

Do pracy z tymi bibliotekami należy je zaimportować, korzystając z poniższego kodu:

\begin{lstlisting}[language=Python]
import tensorflow as tf
import cv2
import matplotlib.pyplot as plt
import numpy as np
\end{lstlisting}

Dzięki temu mamy dostęp do wszystkich funkcji i klas oferowanych przez te biblioteki, co pozwala na efektywną realizację projektu.

Kolejnym krokiem jest konfiguracja pamięci GPU, aby uniknąć wykorzystania całej dostępnej pamięci przez TensorFlow. Dzięki temu inni użytkownicy również będą mogli korzystać z GPU.

\begin{lstlisting}[language=Python]
import tensorflow as tf
import os

gpus = tf.config.experimental.list_physical_devices('GPU')
for gpu in gpus:
tf.config.experimental.set_memory_growth(gpu, True)

tf.config.list_physical_devices('GPU')
\end{lstlisting}

\section{Wczytywanie danych}
W celu wczytania danych do modelu TensorFlow, należy zacząć od zorganizowania katalogów ze zdjęciami. Klasyfikacja obrazów odbywa się na podstawie nazw folderów, gdzie każdy folder odpowiada jednej klasie. W opisywany, przypadku, są dwa foldery: 'uszkodzony' i 'prawidłowy'.

Podczas wczytywania danych, należy skontrolować jakość zdjęć, sprawdzając, czy można je załadować do biblioteki OpenCV oraz czy ich rozszerzenie jest zgodne z oczekiwanymi (np. JPEG, JPG, BMP, PNG). W przypadku wykrycia uszkodzonych lub nieobsługiwanych obrazów, są one usuwane. Poniżej przedstawiono fragment kodu odpowiedzialny za kontrolę i usuwanie wadliwych zdjęć:

\begin{lstlisting}[language=Python]
for image_class in os.listdir(data_dir):
for image in os.listdir(os.path.join(data_dir, image_class)):
image_path = os.path.join(data_dir, image_class, image)
try:
img = cv2.imread(image_path)
tip = imghdr.what(image_path)
if tip not in image_exts:
print('Zdjecie posiada nieobslugiwane rozszerzenie {}'.format(image_path))
os.remove(image_path)
except Exception as e:
print('Wystapil problem ze zdjeciem {}'.format(image_path))
\end{lstlisting}

\section{Ładowanie danych}
W kolejnym kroku trzeba załadować dane za pomocą funkcji \texttt{image\_dataset\_from\_directory} z biblioteki TensorFlow. Funkcja ta automatycznie wczytuje zdjęcia z katalogów i przetwarza je do odpowiedniego formatu. Następnie konwertujemy dane na iterator, aby uzyskać dostęp do poszczególnych zdjęć i ich etykiet.

\begin{lstlisting}[language=Python]
import numpy as np
from matplotlib import pyplot as plt

data = tf.keras.utils.image_dataset_from_directory('data')
data_iterator = data.as_numpy_iterator()
batch = data_iterator.next()

fig, ax = plt.subplots(ncols=4, figsize=(20,20))
for idx, img in enumerate(batch[0][:4]):
ax[idx].imshow(img.astype(int))
ax[idx].title.set_text(batch[1][idx])
\end{lstlisting}

\section{Wstępne przetwarzanie danych}
Wstępne przetwarzanie danych polega na przygotowaniu danych wejściowych, tak aby model uczenia głębokiego mógł lepiej zrozumieć i wykorzystać informacje zawarte w tych danych. W przypadku tego projektu, wstępne przetwarzanie danych obejmuje przeskalowanie wartości pikseli do zakresu 0-1, co ułatwia uczenie się modelu.

Przeskalowanie wartości pikseli polega na podzieleniu wartości każdego piksela przez 255, co jest najwyższą wartością możliwą dla wartości piksela w formacie RGB. Wynik tego przekształcenia daje wartości zmiennoprzecinkowe w zakresie od 0 do 1, co jest preferowanym zakresem wartości dla wielu algorytmów uczenia głębokiego, w tym również dla sieci neuronowych.

W poniższym fragmencie kodu przeskalowanie wartości pikseli jest wykonywane na całym zbiorze danych za pomocą funkcji \texttt{map}. Funkcja ta przekształca dane przy użyciu podanej funkcji, w tym przypadku przeskalowując wartości pikseli dzieląc przez 255.

\begin{lstlisting}[language=Python]
data = data.map(lambda x, y: (x / 255, y))
data.as_numpy_iterator().next()
\end{lstlisting}

Przeskalowanie wartości pikseli ma kilka zalet. Po pierwsze, przekształcone wartości są mniejsze, co może przyspieszyć proces uczenia się modelu. Po drugie, przeskalowanie wartości pikseli może również przyczynić się do poprawy zdolności generalizacji modelu, ponieważ wartości w podobnym zakresie mogą być łatwiej porównywalne przez sieć neuronową.

\section{Podział danych na zestawy treningowe, walidacyjne i testowe}
Podział danych na zestawy treningowe, walidacyjne i testowe jest kluczowym elementem procesu uczenia się modelu głębokiego. Różne zestawy danych mają różne funkcje w procesie uczenia się:

\begin{itemize}
\item \textbf{Zestaw treningowy} służy do uczenia modelu. W trakcie uczenia model aktualizuje swoje wagi na podstawie błędów, które popełnia w prognozowaniu etykiet dla danych treningowych.
\item \textbf{Zestaw walidacyjny} służy do monitorowania postępów modelu podczas uczenia. Na podstawie wyników na danych walidacyjnych, możemy dostosować parametry modelu, takie jak liczba epok, funkcje kosztu czy hiperparametry optymalizatora. Zestaw walidacyjny pomaga również w wykrywaniu problemów, takich jak nadmierne dopasowanie (overfitting).
\item \textbf{Zestaw testowy} służy do oceny końcowej wydajności modelu po zakończeniu uczenia. Wyniki na danych testowych dają nam informacje na temat zdolności generalizacji modelu do danych, których nie widział wcześniej.
\end{itemize}

Aby podzielić dane na odpowiednie zestawy, najpierw należy określić ich rozmiary. W tym przypadku, 70\% danych zostanie przeznaczone na trening, 20\% na walidację i 10\% na testowanie. Następnie trzeba skorzystać z metod \texttt{take}, \texttt{skip} i \texttt{take} dostępnych dla obiektu dataset w TensorFlow, aby wydzielić odpowiednie części danych.

\begin{lstlisting}[language=Python]
train_size = int(len(data).7)
val_size = int(len(data).2) + 1
test_size = int(len(data)*.1) + 1

train = data.take(train_size)
val = data.skip(train_size).take(val_size)
test = data.skip(train_size + val_size).take(test_size)
\end{lstlisting}

Warto zauważyć, że podział danych może być również wykonywany za pomocą funkcji \texttt{train\_test\_split} z biblioteki scikit-learn, jednak w tym przypadku korzystamy z funkcji dostępnych w TensorFlow.

Wybór odpowiedniego podziału danych zależy od wielu czynników, takich jak liczba dostępnych danych, złożoność problemu oraz złożoność modelu. Zbyt mały zestaw treningowy może prowadzić do niedouczenia modelu, podczas gdy zbyt mały zestaw walidacyjny lub testowy może prowadzić do niedokładnej oceny wydajności modelu.